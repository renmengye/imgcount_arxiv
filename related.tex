\section{Related Work}

Instance segmentation has recently received a burst of research attention, as
it provides higher level of precision of image understanding compared to object
detection and semantic segmentation.

\textbf{Instance segmentation using graphical models}. An early exploration of
instance segmentation proposes a multi-stage pipeline composed of patch-wise
features based on deep learning, combined into a segmentation tree
\cite{silberman14insseg}. They formulated a new loss function, the coverage
score, that calculated the amount of ground truth regions not covered by the
model's instance segmentation. More recently, Zhang et al. \cite{zhang16insseg}
formulated a dense CRF for instance segmentation; .  They apply a CNN on dense
image patches to make local predictions, and constructed a dense CRF to produce
globally consistent labellings. Their key contribution is a shifting-label
potential that encourages consistency across different patches. They achieved
strong results on the challenging KITTI object dataset; however, the graphical
model formulation entails long running times, and their energy functions are
dependent on instances being connected and having a clear depth ordering.

\textbf{Instance segmentation using CNN}. Liang et al.~\cite{liang15pfinsseg}
used a CNN to generate pixel-level object size information, and used clustering
as a post-processing step. They added a regressor at the top of the CNN to
estimate the count, which is the total number of clusters. An erroneous count
can usually leads to poor segmentation. Dai et al.~\cite{dai15insaware}
proposed a pipeline-based approach and won the MS-COCO instance segmentation
challenge. Their method first predicts bounding box proposals and extracts
regions of interest (ROI), then uses shared features to perform segmentation
within each ROI. Their architecture can also be fine-tuned end-to-end. However,
since their method is based on detector proposals, it does not explicitly
handle object occlusions, which may lead it to fail during  non- maximal
suppression (NMS). Uhrig et al.~\cite{uhrig16insseg} presented another approach
with FCN, and achieved very impressive results. Their FCN outputs three
channels: semantic segmentation, object orientation and depth. Post-processing
based on template matching and instance fusion produce the instance identities.
Their approach is based on bottom-up clustering since their FCN can only
provide pixel-level information, whereas our model is processing the image in a
top-down fashion. Importantly, they also used ground-truth depth labels in
training their model.

\textbf{Instance segmentation using RNN}. Another recent line of research, e.g.
\cite{stewart15lstmdet, park15ris, romeraparedes15ris} employs end-to-end
recurrent neural networks (RNN) to perform object detection and segmentation. A
permutation agnostic loss function based on maximum weighted bipartite matching
was proposed by \cite{stewart15lstmdet}. To process an entire image, they treat
each element of a $15\times20$ feature map individually. Similarly, our box
proposal network also uses an RNN to generate box proposals: instead of running
the image 300 times through the RNN, we only run it once by using a soft
attention mechanism \cite{xu15caption}. Romera-Paredes and Torr
\cite{romeraparedes15ris} use convolutional LSTM (ConvLSTM)
\cite{shi2015convlstm} to produce instance segmentation directly. However,
since their ConvLSTM is required to handle object detection, inhibition, and
segmentation all at the same time on a global scale, the final output loses
precision. They add a dense CRF to restore the resolution. Compared to their
approach, our segmentation network operates on a local level. Instead of
resorting to graphical models, we added skip connections to restore the
resolution.

\textbf{Instance counting}. Previous work on object counting in images has
mainly focused on crowds of pedestrians and biological cells
\cite{lempitsky10count}. Chattopadhyay et al. \cite{chattopadhyay16count}
focused on counting questions in VQA and proposed detector approaches as well
as a regression based method (``associative subitizing'') that works on a $3
\times 3$ field of CNN features level. Note that unlike our approach, this
method does not provide instance segmentations.
